This chapter provides essential background information for understanding the subsequent sections of the report, focusing on X-HEEP, HEEPocrates, PYNQ-Z2 FPGA, and the porting of DOOM to embedded systems.

\section{X-HEEP and HEEPocrates}

X-HEEP is an open-source RISC-V microcontroller known as the eXtendible Heterogeneous Energy Efficient Platform. It offers extensive configurability for various hardware implementations. X-HEEP can be fully simulated in software, emulated using FPGAs, and since 2024, implemented as an ASIC known as HEEPocrates \cite{xHeepWebsite}.

HEEPocrates is a 65nm ASIC implementation of X-HEEP equipped with an Ibex core, 256kB of SRAM, and several peripherals. Designed for power efficiency, it is particularly suitable for biomedical wearables due to its compact size (see Figure \ref{fig:HeepocratesMonnaie}) and low power consumption. The ASIC can operate at clock speeds of up to 470MHz \cite{heepocratesWebsite}.

\begin{figure}[ht]
  \centering
  \includegraphics[width=0.48\textwidth]{images/HEEPocrates_monnaie.png}
  \caption{HEEPocrates chip next to a 2 CHF coin}
  \label{fig:HeepocratesMonnaie}
\end{figure}

\section{PYNQ-Z2 FPGA}

X-HEEP can be fully implemented on the PYNQ-Z2 FPGA board. The current prototyping of X-HEEP into this FPGA is running at 15MHz with up to 512 kB of RAM. Despite its lower clock speed compared to HEEPocrates, it offers twice the RAM capacity.

\section{DOOM on Embedded Systems}

DOOM, released in 1993, revolutionized gaming as one of the first FPS (First Person Shooter) games. It introduced the DOOM Engine, capable of rendering 3D graphics with non-orthogonal walls and floors. The game's source code and a demo level were released as open source, allowing enthusiasts to modify and create new levels and port the game to various platforms. DOOM separates code from game data, with all levels stored in \texttt{.wad} files containing textures, sounds, and other game assets. During gameplay, level data is loaded into RAM, and graphics are rendered in a screen buffer with a fixed resolution of 320x200 pixels. The original DOOM code is optimized for 32-bit CPUs without floating-point operations, aligning well with X-HEEP's architecture.

DOOM has become a cultural touchstone with the meme "Does it run DOOM?" inspiring communities to find new devices capable of running the game \cite{canItRunDOOM, redditDOOMCommunity}. \\

Two notable ports include versions for the Nordic Semiconductor nRF5340 \cite{nordicDOOM} and EmbeddedDOOM \cite{embeddedDOOM}, a lightweight version optimized for low-RAM devices and running on Linux. The theoretical absolute minimum RAM requirement of EmbeddedDOOM is still 384kB, not counting the \texttt{.wad} file, which is still too much by a factor of 1.5 for HEEPocrates.

For this project, the Nordic DOOM variant served as the foundation for porting DOOM to X-HEEP. Leveraging features like storing WAD files in flash memory and loading data on demand, significant reductions in RAM usage were achieved. Notably, even the smallest public demo level occupies 4.2MB of space.
